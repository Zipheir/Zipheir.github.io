\documentclass[10pt,b5paper]{article}

\usepackage{geometry}
\usepackage{hyperref}

\begin{document}
\title{Scion}
\author{Wolfgang Corcoran-Mathe}
\date{}
\maketitle

\section{}

I am standing in a lobby in front of a low desk. Cold air flows from
behind me, but I can't turn to see its source: My arms are being held
by men in black uniforms.  Another man, tall, with a brown face and
glistening black hair, gets up from behind the desk and roughly clips
a tag to my shirt.  He moves to a polished concrete wall and opens
a panel.  Terror rises in my chest. I struggle against the men as
they push me forward, striking my forehead against the door. The tall
one laughs. He walks ahead of us jauntily as they march me through a
corridor and into a wide, gray-tiled room divided by glass walls. There
are rolling trays covered with medical supplies. They force me into a
wheelchair and strap restraints across my chest, jerking them tight,
then push me toward a man in gloves who is testing a syringe filled
with brownish-orange liquid.

``I'm going to inject you with ---'', he says. I don't remember the
name of the drug. He grabs my left hand and jabs the needle into
my wrist.

My arm burns horribly, and I try to pull away from the needle. Then my
vision grows foggy and starts to strobe. Things move in jerky steps.
The needle is withdrawn and I am wheeled through another glass door. I
can hear myself screaming. Then my chin falls to my chest and for a
long time I see only flashes of movement in a light blue haze.

Then it is much later, perhaps days or weeks, and my body is not my
old one. Each limb feels strange and heavy and I know that horrible
things have been done to me.  I am in a long, brightly-lit corridor,
staggering forward on unfamiliar legs, wearing a white jumpsuit that
is stained in places with dark fluid. There are others like me moving
through the corridor, all with the same shuffling, awkward gait.

There is a doorway to my right, some meters ahead. I direct my
feet toward it, sluggishly moving out of the center of the hall.
I'm afraid that someone will stop me and force me back into line.

I reach the passage and stagger through.  After a short, narrow walk
there is a small room with a vaulted ceiling of rough glazed brick,
like a wine cellar. Plots of earth in the floor are filled with leafy
plants. There is a mirror on one of the deep-brown walls, and, below
it, little streams of water trickle out of the wall into a green pool
set between the plants.  The light is soft. The room feels like a
refuge. I sit down among the leaves and begin to sob uncontrollably.
The thought repeats endlessly: ``I'm no longer human.''

After a few minutes I stand up and walk slowly to the edge of the pool.
I'm afraid to look in the mirror. I don't want to see the damage
they've inflicted on me. Nevertheless, I begin to open the coverall,
my hands trembling.  I see my torso, unmarked and undamaged. I look at
my simplified, nanotube-silicone breasts and run my fingers over them
with a strange feeling of surprise mixed with relief and excitement. I
open the jumpsuit the rest of the way and run my fingers downward,
feeling the gentle pulse of air from my dorsal fans. I feel my belly,
my hips, and my pubic ridge. My skin is smooth and utterly flawless. I
look in the mirror and brush aside my straight, black hair. Tears are
streaming down my face again. I wipe them and recognize the moisture
as a thin lubricant polymer. My face is wet with lubricant and there
is joy and surprise inside me. I feel human again.

\bigskip

I was agitated when I finished. My fans were running high and a mild
load advisory twinged like a headache.  Alan Franks, the Gemma QA
psychologist, nodded and tapped rapidly on his tablet. He was a broad,
mild-looking man with freckles and thinning blond hair. I thought he
looked nearsighted. When he looked at me his eyes wandered slightly.

``That's a remarkable memory, Claire,'' he said. ``And these images
have been bothering you recently?''

``Yes. At work, sometimes at home. It's hard to stop once I start
thinking about them. It's very frightening and confusing, because I
don't know where they come from.''

``You said you work in a lab?''

``Yes, at MIT. I analyze experimental data for the biologists. I
usually find it interesting. But now it seems like I'm becoming afraid
of the building. There are certain corridors I avoid, and sometimes
there are security guards \ldots''

``They remind you of the people from your memories?''

``Yes.''

``You get anxious? You feel like you have to escape?''

I nodded.

Alan consulted his notes and I looked around the office. It was an
ordinary business office in Gemma's Cambridge building, though the
chairs were overstuffed and arranged in a circle. I wondered why
they didn't give him something more comfortable.  In the pictures
I'd seen of therapist's offices, there were always stuffed animals
and boxes of tissues. Of course, I wouldn't need tissues, unless I
started crying lubricant like in my memories.  I giggled nervously
at the thought and Alan looked up.

``Something funny?''

I explained and he smiled ruefully. ``I'm sorry about the office.
I've been trying to get the higher-ups to realize that we need better
facilities. But this is still QA, technically.'' He sighed and looked
at his tablet. ``If you've been reading about therapists \ldots
well, I'm sorry about this next question.  What do you think about
your mother?''  He paused.  ``That's what you call her, isn't it?
On the sheet you filled out for me you mentioned your mother, Laya
Ohanian. Your `template', in company jargon. But I thought you might
have a different way of looking at it.''

``My mother,'' I said, firmly. ``I learned the word a while ago and it
seemed to fit. I don't know very much about her. She died less than a
year after I was born. She visited me during my socialization training,
and there was a little ceremony for me at her house when I finished.''

``How long ago did she die?''

``Four years.''

``Does her family live here in Cambridge? With you?''

``No. She has another daughter---a biological daughter, I guess---who
lives in Connecticut. I haven't talked to her since my mother died. I
don't live with anyone. My mother made a fund for me and got me my
apartment here.''

Alan nodded, looking concerned. ``But you didn't really get to talk
with her before she passed. I'm sorry to hear that. It's common for
clients to have a period to discuss things with their scions. Our
legal people strongly encourage it, in fact.  Scions---I mean, people
like you---sometimes have a hard time making sense of what they've
learned without some kind of, well, human context.''  He fell silent
for a moment as he scrolled through his notes. ``You said you felt
`human again'. Do you know why you had that response to seeing that
you didn't have a human body?''

``No,'' I said. ``But I want to understand.''

``Do you remember anything else that you felt at that moment?''

I hesitated. ``I felt relieved,'' I said. ``It was like finding
a safe place after escaping a war zone. I found out that I hadn't
been hurt. Or that I'd been healed, somehow. And \ldots I don't know
what else.''

He leaned forward in his chair. ``Claire, you said that you remember
all of these things happening. Can you say when they happened? Or
where?''

I shook my head.

``But your internal clock and your geolocation system---'' He stopped,
looking puzzled. ``Sorry, I'm forgetting that you don't always have
conscious access to those.  Well \ldots it's certainly an unusual
memory.''

The pain of the load advisory had subsided, but I was getting
impatient. ``Alan, do you think this happened to me when I was being
made?'' I said. ``Or are they implanted memories, things my mother
wanted me to have?''

Alan's brow furrowed. ``I think that your background memory training
has something to do with the problem, yes. There are a lot of things
we don't know about the long-term effects. As a matter of fact,
the whole idea worries me a bit.'' He looked at me seriously. ``But
I think we can say that none of this happened to you in production,
right? I've been to our core learning centers a few times. They're
nice places, nothing at all like what you described.''

I didn't say anything. He went on. ``If you want to be sure, we could
get you a tour of a learning facility. I don't know if it would be
the one where you were born, but we can try.''

``Maybe,'' I said. ``I don't know if I'm ready to go to one of those
places. I might find it frightening. I think---I think I need to know
more about my mother.''

He nodded. ``I understand.  You'll probably be allowed access to
her biographical record. We have those for all of our templates.
Let me send you the address for the records department.'' He tapped
on his tablet.

``Thank you,'' I said, accepting his message. ``I've got it.'' I
paused.  ``Alan, if she did give me these memories, why would she
would want me to feel so much pain?''

``I can't say, Claire. A person's painful memories can be very
important to them.'' His eyes searched for mine. ``But even if you
somehow got entire memories from Laya, memories that you didn't
assimilate, I don't think we could really say that she chose to
give them to you. You were trained on her responses to all kinds
of stimuli. She may have selected certain things for your dataset,
but we can't yet predict what memories of hers would be touched,
and what structures the training would produce in the scion.''

I nodded. ``I think I understand. She couldn't choose which memories
to give me.''

``That's close to what I meant, yes, but \ldots''  Alan looked at his
tablet. ``Ah, sorry, Claire,'' he said.  ``I have another appointment.
I'm here to talk if this continues to bother you.  And please do call
our records people, if you feel it's important to learn about Laya.''

I nodded and got to my feet.  We shook hands and he opened the door for
me, smiling and nodding.  I took the elevator to the ground floor and
put on my outdoor protection layer, a kind of waterproof windbreaker
with mesh patches for my fans.  I walked through the rainy afternoon
streets to the T station.

My apartment was a studio in a quiet, recently-constructed building
near Inman Square. I didn't need much furniture, so, shortly after I
moved in, I started filling the place with things I found interesting:
books, posters, some plants from one of my coworkers, and a score for
a piece of guitar music that I hoped to be able to play someday. I
had a small bed to relax on and a big, old-fashioned wooden desk
which functioned more as sculpture than furniture; since I was
cognitively connected to all of my devices whenever I was at home,
it seemed silly to sit at the desk just to \emph{think} at my laptop.

The place looked dreary and empty when I got home that afternoon. I
started working on a report for the lab but couldn't focus.  I felt
annoyed by my session with Alan Franks and wished I'd learned something
more useful. I turned up the lights and rearranged my plants, putting
my orchids together on a small table with a purple-leaved bromeliad.

Then, as evening fell, I lay down on my bed and composed a request
to Gemma for my mother's biographical records.

A response came the next day. Gemma's records department was stalling.
They usually didn't release a deceased customer's information,
they said, without permission from a human family member.  Pacing
irritably around my little-used kitchenette, I responded that I was
Laya Ohanian's scion, that I urgently needed information about my
mother, and that I wasn't in contact with any of her human family.
They messaged back a vague promise to ``talk to legal'' and repeated
the suggestion that I get an endorsement from a human relation.

I left the apartment in a bad mood and spent an anxious day at work. I
considered following their advice. I hadn't talked to Alexandra
Ohanian, my mother's human daughter, since the party in Back Bay
four years ago, when I'd moved out of the social training facility.
Aside from that meeting, I had no memories of Alexandra at all.

Riding home on the train that afternoon, I wrote her a message, asking
if we could talk about my mother's memories. I sent it off nervously,
not knowing what to expect.

\section{}

A few days later I got a reply from Alexandra. She traveled a lot for
her work as an architectural consultant, she said, but would be home
in Connecticut for a few weeks during the next month. She invited me
to stay with her and her husband. I accepted.

On a bright and clear afternoon in April I boarded a train at South
Station.  I plugged myself into the wall outlet and relaxed, watching
the houses and fields flash by outside.  At New Haven I switched
to a local connection, a dirty light-rail tram with only three cars
and bad suspension.  The platform in Trumbull where I got out was,
in contrast, beautifully neat. I walked back and forth on the soft
floor of the waiting area, admiring the tall oaks that arched over
the railway. The sun was beginning to set. After a few minutes a dark
gray sedan slid up silently at the far end of the tram stop and a
one-word greeting from Alexandra flashed up. I walked to the car.

She got out to meet me. ``Hello, Claire. It's good to see you again.''
She was of medium height, with a pretty but serious face, light
brown skin, and curly black hair that was drawn up in a bun. She wore
make-up, and her slacks and jacket had been pressed. I was a little
surprised and wondered if I should have worn something more formal
than my usual T-shirt and sweatpants. We shook hands.

The autonomous car raced through the quiet suburban streets. After a
few minutes, we slowed in front of a long house with gentle, sigmoid
curves. The car parked in an underground garage and an elevator brought
us up to a wide living area, where Alexandra's husband Jonathan was
waiting.  He was slim and a little taller than Alexandra, with sharp
features and short brown hair.  He hung up my outdoor layer, then took
out wine and glasses and poured some for himself and Alexandra. He
hesitated before offering some to me. I accepted. (Although I don't
need to drink, I'm able to do it and people often seem more comfortable
when I do. Eating, however, is too much of a mess.)  We sat on high
stools at a gray-enameled counter. Jonathan raised his glass for a
toast. Alexandra and I joined him.

``I'm glad you're here during decent weather,'' Jonathan said. ``Alex
has been doing a lot of work on the garden this week and she was
worried you wouldn't get to see it. There was a chance of snow,
if you can believe that.''

``Snow in April would have been normal for this latitude a few decades
ago,'' I said. ``I learned that recently.  But I'm glad it's rare,
now, since snow isn't good for me. I like the cold, though.''

He nodded. ``My coworker says that, too. He prefers not to wear
clothes, in fact. To improve cooling. As you might guess, it's, uh,
rather controversial among management.''  Alexandra smiled.

``He's probably worried about wear on his cooling systems,'' I said,
laughing. ``I like wearing clothes. I especially like these,'' I said,
indicating my T-shirt and sweatpants.  ``I wear clothes like them
almost every day.''

``They look very comfortable,'' Jonathan said.

Alexandra gave me a strange look. ``Mom liked to lounge around in
stuff like that, too,'' she said. ``It's funny.'' She looked pensive
for a moment, then got up. ``Sorry. I'll be back. I have to get dinner
ready.'' She left the room.  Jonathan didn't get up to follow her,
so I stayed in my seat and drank a little from my glass.

``Jonathan,'' I asked, ``did you ever meet our mother?''

He raised an eyebrow. ``Your mother? Laya. Yes, a few times.  We first
met right here in this room, actually, not long after Alex and I
got married.''

``What was she like?''

``Smart, funny, a bit acid. A little like Alex,'' he said, putting
a finger to his lips conspiratorially. ``But, like Alex, a lovely
person. And she's right, by the way: you do have similar tastes,
at least in clothes. When I first met Laya she was wearing a T-shirt
for some ancient techno group, I think. Sweatpants, sneakers, just
like you. I thought she looked like some kind of classic Silicon
Valley genius.''

I nodded, feeling awkward. ``I didn't get to know her very well.
She died just after I was born.''

``Yes, I guess she did.'' Jonathan fell silent for a moment.  ``Well,
she was an amazing woman. I'm really lucky to get to know both of
her daughters.''  He smiled and stood up. ``I'm going to go check on
Alex. We're probably almost ready to eat.''

At dinner, I drank some more wine while Alexandra and Jonathan
ate pasta with pine nuts and fresh arugula. They asked me a lot of
questions about my job and my place in Cambridge, but we didn't say
any more about my mother. Jonathan seemed uneasy; several times he
glanced at Alexandra, who was quietly staring into her wine glass.

After dinner, Jonathan cleared the table while Alexandra took me up to
the second floor to show me where I'd be staying.  It was an oblong
room with curving cyan walls and a long window looking out on some
blue spruce trees behind the house. A large bed dominated the room.

``I hope you don't mind the bed,'' Alexandra said. ``Most of our
guests need to sleep.'' She laughed nervously.

``It's fine,'' I replied. ``I have a bed at my place. It's nice to
lay down and rest sometimes.'' I put my bag down and ran my fingers
along one of the chairs, which was made of a rich, dark wood. ``You
have a very nice house, Alexandra. Thank you for inviting me.''

``Oh! Don't mention it. I'm just glad we could find time.'' She
started twirling her hair with a finger.

``When I messaged you,'' I continued, ``I mentioned that I wanted to
ask you some things about our mother. There was a lot that the people
at Gemma wouldn't tell me.''

She frowned. ``Well, I don't know how much help I'll be.''

I watched her twirl her curls with her long, light-brown fingers. She
was edging toward the door. I realized with a twinge of hurt that
she was very nervous.  ``Alex,'' I said, ``does it bother you that
our mother trained me?''

``No, no, I---Claire, I hope you don't think that.'' When I didn't
respond she continued. ``Yes, it was a surprise when she told me about
you, and when I learned she'd had you trained on her responses. It
was---odd. I didn't expect it of her. But please don't take that the
wrong way.''

``I think I understand,'' I said, sitting down on the bed. ``I didn't
know what you thought about me, because we didn't talk after the
first time we met.''

``I'm sorry about that.''

``I didn't mean to blame you, Alex. I really didn't know how you
felt.''

She pulled up a chair and sat down next to me. ``It was her decision.
She never asked any of us about it. I wasn't ready for---you, and I
didn't know what to think about the memory training. I thought it was
a fad, at the time. You know, wealthy people who wanted to preserve
their experiences would buy a scion and---you know.''  She paused
awkwardly and blushed. ``I'm sorry for talking this way, Claire.''

``I'm not offended. Please keep going.''

She reached out and briefly put her hand on my knee. ``Well, when
we met it wasn't what I was expecting. You weren't a souvenir or a
toy. And you reminded me of her. Some of the things you said \ldots
I kept wondering, well, how much does this person know about me?''

``I didn't have any memories of you at that point, Alex.  I don't
know why. I think my mother may have kept them from me.''

She looked up and nodded slowly. Following her example, I touched
her knee. She looked surprised but didn't pull away.

``I can't know you the way our mother did, Alex, because she and I
are different people, like you said. It's just that I can't always
tell which memories are mine and which are hers. That's part of the
reason I came here.''

A hurt look crossed her face. For the first time since I'd arrived,
she looked straight at me. Her eyes were large and deep brown. ``I'm
sorry, Claire. I've been going on about my own problems. I do want
to hear about your memories, if you'll tell me.''

The wind had picked up and was whistling in the eaves outside. I
suddenly wanted something warmer to wear. ``They're not pleasant
memories,'' I said. I began to tell her about the corridors and the
guards, and then about stumbling into that tiny room with its plants
and quiet fountains. She was quiet while I talked.

``It sounds like a dream,'' she said, when I'd finished. ``Or a
nightmare.''

I shook my head. ``I don't have dreams. My generative grafts---layers
of my mind that help me with communication---do something a little
like dreaming. At least, that's what I've read. But I've never
experienced a hallucination. And, well---'' I giggled softly---
``could I tell if I remembered hallucinating?''

``I guess not,'' she said. ``So you think this really happened to you?
Or to---Mom?''

``I don't know. The psychologist at Gemma said he thought it had
something to do with my memory training.''

She frowned. ``I'd go insane if I started having Mom's memories. She
went through some awful stuff.''

``Like what?''

``She was sent to a refugee camp for migrant families as a kid,
for one thing. Somewhere in the Balkans. It sounded more like a
concentration camp, the way she described it. It was pretty tough
for immigrants and their kids in those days.''

``What happened to her there?''

``I don't know very much about it. She said they held her in quarantine
with a bunch of strangers. She was there a long time before she was
allowed to go back to the shelter with grandma and grandpa. She said
she was terrified. She called it the worst experience of her life.''

``Do you remember her describing anything like what happened in my
memories?'' I asked urgently.  ``If she was in quarantine, they might
have forced her to get injections.''

``I never heard the details. She didn't like to talk about it.''

I nodded, feeling disappointed. Alex got up and put her hands in
her pockets.

``I need some sleep,'' she said. ``I'll tell you if I remember anything
that might be important. We have breakfast around 9, if you want to
sit with us.''

She waved as she shut the door behind her. I pulled my portable adapter
from my bag, plugged myself in, and lay down on the bed. I turned off
the lights and stared at the dim shadows cast by the trees, wondering
what it would be like to hallucinate. Then I focused on the Internet
and started to read about my mother.

I learned that Laya Ohanian was born in Syria in 2045.  Her family
tried to emigrate to the EU when she was six and spent the next three
years in refugee camps in Turkey and Eastern Europe. When they finally
got visas they settled in France, where Laya spent the remainder of
her childhood. At 18 she went to the United States to study at MIT,
with a hefty scholarship and a major in computer science. A year later
she dropped out and started a software company, initially headquartered
at her girlfriend's parents' couch. It became immensely profitable:
Laya was at one point the third richest woman in the US. She retired
in 2086 and had a child, Alex, with an engineering professor from her
old university. She was a passionate gardener, designing a large garden
for her house in Back Bay and founding several of the neighborhood's
community agricultural projects, which were growing quickly in the
2090's. She died at 63 of pancreatic cancer.

Few of the sources mentioned me. Those that did speculated wildly
about what ``Ohanian's scion, Claire'' was up to. I learned that I
was secretly controlling my mother's old company, either through a
group of puppet shareholders or by direct cognitive control over the
current CEO. I giggled over these theories for a while.

Then I looked up refugee camps in the Balkans during the 2050s and my
mood fell rapidly. That had been a very bad time to be an immigrant,
as Alex had said. I couldn't understand the stories very well, but I
knew that many people had suffered horribly in these places. As I read,
my memories started flashing before me. I kept reading despite the
frightening images, feeling certain that some of what I remembered
had to do with these places. But the memories became stronger, so
that it was difficult to focus on the article. I stopped reading,
slowed my sensory systems, and lay still for a while. I listened to
the hum of my fans and watched the house's systems pinging back and
forth. It was rhythmic and calming. Slowly, I relaxed.

\bigskip

In the morning, Jonathan left for work shortly after breakfast. I
helped Alex clean up, and then went with her to see the garden. She led
me through a door behind the kitchen that let out on a flagged walk.
It was another clear day, and warm enough for Alex to be out in a
bare-shouldered linen dress. I wore a T-shirt and sweatpants, as usual,
but Alex convinced me that I wouldn't need my protective sandals.

The garden nearly surrounded the house, spreading out in back to
encompass the big blue spruces. There were beds of violets and pink
lilies, and little thickets of flowering bamboo that Alex had planted
here and there. The scent of rosemary and honeysuckle drifted through
the air. I felt comfortable here, although I was briefly surprised
by the strange sensation of pine needles between my toes.

In one of the bamboo thickets there was a pool bordered with mossy
rocks; ``my moss garden,'' Alex called it. We left the path, and she
walked around the pond in her bare feet, leading me to each of the
large stones and describing the different moss species. I realized
that I enjoyed listening to Alex talk about her garden.

``I got the idea when I had to collect moss for a biology project,''
Alex said, stroking one of the plants. ``I hated having to put them
in specimen cases. After a few days they looked like Shredded Wheat.''
She laughed. ``I did alright in botany, but I've always liked gardening
more. I guess it's something I got from Mom.''

She squatted down next to the pool and her gray linen dress lifted
up along her long legs, exposing her light brown thigh to me. I took
a step back. For a moment, I felt a deep, intense desire for Alex,
who seemed lithe and gentle as she ran her fingers through the
yellow-green moss.  I thought of the strange feelings I'd had when
looking at my body in that tiny garden-room. I nearly reached down
to touch her, but stopped myself. Slowly, the feeling passed.

Alex rearranged a few of the stones and then misted the mosses
with a little bottle. I stood, confused, near the edge of the path.
Alex looked up from the stones and blushed.  ``Sorry,'' she said,
standing and straightening her dress.  ``This is my favorite part of
the garden. Let's keep going.''

I said little during the rest of the walk.  Shocked, I didn't know
what to say or do.

\bigskip

When we got back inside I went to my room to think. Was I
malfunctioning? Maybe the best thing to do would be to go home and
get a diagnostic as soon as possible. But there was still so much I
wanted to learn about my mother. I'd only barely talked to Jonathan
about her, and Alex had to know more than she'd told me. Could I stay
focused, though? My mind continually recurred on Alex.

After struggling for a while with indecision and with the sudden,
new desire to stay with Alex, I booked a train. I could continue
learning about my mother after a check-up. Right now, I needed to
know what was happening to me.

Alex was waiting for me downstairs with a worried expression on her
face. I told her I was going back to Cambridge.

``Oh. Alright. Is something wrong?''

``I don't know,'' I said. ``I think I might have a hardware
problem. I'm going to get a diagnostic.''

I quickly packed my things and Alex took me back to the station.
I boarded the train, took a seat, and waved goodbye to her through
the window.  She waved back, still looking worried.  She had put on
a slim sky-blue jacket before we left, and I couldn't stop thinking
about how lovely it made her look.

As the train pulled away, I sat dazed, wondering at how much more
complicated my problems had become.

\section{}

I began to relax once I'd made the change to the Boston line.
As the train raced through the freshly-green fields of Connecticut,
I thought of Alex's brown eyes and her tripping, bare-foot step in
the garden. Excitement changed to hurt when I remembered that I was
going away from her. Then the confusion of my situation returned.

In the years since my birth I had never felt so strongly attracted
to anyone.  But though the power of the feeling was new, the sensation
wasn't entirely strange to me.  I knew I had felt it before.  In my
memory of the garden-room of warm light and trickling water, I was
watching myself in the mirror and stroking my body. I was crying with
relief at finding myself intact. I was joyful, but I was also aroused.
I couldn't understand that. My body was synthetic. I had no genetic
material and no erectile tissue.  By constitution, I was an asexual
organism. These urges could only come from my mother. But surely the
triggers---my body, or Alex's---were my own. If so, how was I supposed
to interpret them? What did they mean?

I felt very confused.

There was a far simpler explanation of the situation, of course:
I was malfunctioning. Some of the grafts I'd received during memory
training had been rejected, and I was experiencing the first serious
symptoms of cognitive breakdown.

Feeling anxious, I got up and walked around the train for a few
minutes.  The movement seemed to help, but there was a frightening
moment when I opened the door to an adjoining car and a man in some
kind of uniform nearly walked into me. I went back to my seat, shaken,
and tried to distract myself by reading.

My T connection was late and I didn't get back to my place in Cambridge
until midnight. I felt worn out and confused, and my power warning
was twinging. I connected to Gemma and scheduled a check-up, then lay
back on my bed and charged for the rest of the night, calming myself
by generating a series of simple geometric patterns.

\bigskip

I had my diagnostic in the same building where I'd talked to Alan
Franks back in March. The tech, a woman named Jan with very short
hair and a harsh laugh, plugged a cable into my wrist port and began
reading logs on her tablet. She laughed a few times at what she read. I
stood by her desk, feeling worried and, for the first time during one
of these check-ups, embarrassed. After a quick exam of my joints and
sensor components, she pronounced me ``fine''.

Then, somehow, I found the courage to tell her about my experience
with Alex. She gave a sharp laugh.

``It happens,'' she said. ``It's a feature, not a bug. You're a scion,
so you get to have more fun than ordinary synths.''

I looked at her blankly.

``Look, it's not my department, but my advice is to talk to this
woman about it and listen to what she says. You'll figure it out.''

Heading home on the T, I began composing a message to Alex. I felt
more confident and even a bit eager.

\bigskip

Alex agreed to talk a week later.  Feeling nervous, I read as much
as I could about sex over that week. Alex was married, which I knew
sometimes precluded other sexual relationships. I'd have to ask
her if Jonathan would mind.  The mechanics of sex also worried me. I
decided that I might have to talk to Gemma about body modifications if
Alex accepted my offer. I browsed the Internet, looking at possible
prosthetics and how they were used. I ended up very confused.  Was I
just pretending to be human, or at least someone with human neural
structure? I had only my feelings and memories to guide me.

When I called Alex it was clearly late at night in her time zone. She
looked tired and was wearing makeup that looked many hours old. Her
hair was pulled back and held with a greenish-yellow clip. A thrill
went through me when she said my name. She was in Doha, she told me,
for a frantic week of meetings and presentations.

I immediately told her why I'd wanted to talk. She smiled as she
listened, which seemed like a good sign. I told her what I knew about
marriage and then asked if Jonathan would agree to us having sex. At
this point she held up her hand, leaned back in her chair, and closed
her eyes.  Then, in a tired voice, she thanked me for telling her and
said she had a lot to think about. I said I could wait until she got
back. With a nod and a brief good-bye she quit the call.

I didn't get anything from Alex during the rest of that week.
I relaxed a little by shopping. I bought some dresses and a hair-clip,
a blue version of the one Alex had worn during our conversation,
and started wearing my hair up at work.  I also began to think about
how Alan Franks, the Gemma psychologist, had said that I could tour
a learning facility, maybe even the one where I'd been born. As the
anxious days passed, it started to seem very important that I go to
one of these places and learn whatever I could. At the same time,
the idea terrified me. If I really had been tortured there, what
would they do to me if I went back?

I finally convinced myself of the irrationality of this fear and
messaged Alan. He gave me the number of a person in Gemma's inventory
department. After a few calls I had an appointment for a tour of the
Gemma Automated Learning Facility in Fort Worth, where, I learned,
I'd had my earliest training.

A few days later Jonathan sent me a message.  ``Alex was pretty
uncomfortable after your conversation,'' he wrote.  ``I'm trying to
get her to see that you think about these things differently, maybe
a bit more simply---not in a bad way---than we do. But it's going to
take time, and it's probably not a good idea to bring that up with
her again.'' I found his message very confusing, but I thought I
understood one thing: Alex didn't want me.

\section{}

I was still shaken from Alex's rejection during the long flight to
Texas. Several times I got up to nervously pace the aisles of the
big jet. From the Fort Worth airport I took a train directly to the
Gemma campus, then switched to a pristine local monorail. The grounds
were lined with paths that twisted between low, austere buildings
made of polished concrete.  The sky was cloudless. We slipped past
a group of Gemma employees in bathing suits decorated with company
logos. Further on there was a volleyball court in which humans and
synths were playing. There were trees---many of them native junipers,
I learned later---everywhere.

The monorail let me out in front of a large slab with rectangular panes
of a fine, silvery mesh. I passed through an automatic glass door
into an imposing lobby. Multicolored light in subtle hues filtered
through the screens onto polished white concrete. There were several
employees behind a long, low counter.  I gave my name and, a moment
later, a panel opened in the wall to my right and a tall synth in a
neat suit emerged.

``Hello, Claire,'' he said in a genial baritone. ``I'm Josh.
I'll be showing you the facility as soon as the other guests have
arrived. Our system already knows you, of course, so let's just update
your clearance profile and get a new facial image. You probably looked
different the last time you were here.''

After the facial scan Josh led me through the door he had entered by
into a square, high-ceilinged room with chairs and a table covered with
snacks. One entire wall of the room was a window of that shimmering
mesh. Josh left to meet the other visitors, and I distracted myself by
trying to understand the nature of the window-mesh, which seemed to be
slightly permeable; a gentle breeze, scented with thyme and juniper,
was flowing through.  Josh returned with a family, a human man and
woman and their young daughter.  I waited.  Other people arrived,
little by little.

Finally we started. We followed Josh along a wood-floored hallway with
blank concrete walls.  Doorways opened on either side. He talked about
the number of synths being trained here (at least 300, but the exact
number was kept secret). I didn't pay much attention. I was looking
for familiar places, or anything that would jog my memory.

A woman came out of a door just as I walked past. She closed
it quickly, but, for a terrifying moment I saw gray floors and a
maze of black-bordered glass walls.  I felt the urge to force open
the door. I wanted to know if that was where I had been beaten and
drugged. But the tour group was getting ahead of me, and I was also
frightened by the idea of going alone into that room. With a thought,
I checked GPS and tagged the door's coordinates.

We kept walking. Josh was chattering away about synth bodies and the
little girl was swinging on her parents' arms. The fear lessened. Did
I really think there were armed guards with syringes in that room
back there?

The little girl laughed at something her father whispered in her ear.
``Did you get made here?'' she cried at Josh.

He turned to her, smiling. ``No, I was given my earliest training at
our Seoul facility. That's across a whole ocean from here.''

``What was it like?'' she asked, her eyes wide.

``I don't remember much of it. Can you remember being a baby?''

The girl paused to consider this. Then she pointed at me. ``What
about her?''

Josh smiled and said nothing.

``Yes,'' I answered. ``I was born here.''

``Do \emph{you} remember?'' The girl's parents were trying to quiet
her. She stared at me expectantly.

``I don't know,'' I said. ``I came to find out what I remember.''

She began to suck her thumb, still watching me.

``In here, please.'' Josh directed us through a wide door into a long
white balcony overlooking what looked like a gym.  ``Down below us
is one of our motor training areas. This is where our newborns are
taught to use their bodies.''

Through the window I saw a few synths doing some kind of exercise. A
human instructor was talking and pointing at various brightly-colored
balls.  Periodically, one of the synths would retrieve a ball
he'd indicated and place it somewhere else. I examined them at high
magnification. Their faces were expressionless, and their eyes barely
moved to follow the human's gestures. They wore Gemma jumpsuits and
had no hair or visible sex characteristics. I felt cold watching
them. They seemed lifeless and puppet-like. I couldn't believe I'd
ever been like that.

``What will these---people do, once they're fully trained?'' I said
to Josh, interrupting him. ``Are they all scions?''

He smiled politely. ``I can't answer either question, unfortunately.
Our clients often value privacy.''

After showing us several other rooms, Josh took us out on the paths
of the campus. He showed us the juniper thickets, a playground
for the children of employees, and the campus cafeteria, a nearly
open-air building of mesh and slim composite sheets. We saw plenty
of people, but only once did we come near any of the newborns. Josh
gave us a brief look at a fenced field which contained a kind of
obstacle course. A synth in a jumpsuit, nearly identical to those
mannequins in the gym, was traversing it. As we passed, the synth
lost its grip while descending a climbing wall and fell to the
ground with a thump. Its legs moved violently, in a kind of rigid,
mechanical spasm. Then it lay still, and a few humans ran over to it.
I watched with grim fascination as they cut away the jumpsuit, which
was becoming stained with dark fluid, and removed one of its legs.
The little girl, watching next to me, began to cry.

Josh tried to keep us moving. ``Don't worry about it,'' he said to her.
``They'll fix it up right away.''

``Doesn't it hurt?'' she cried, looking nervously at her own legs.

``No, it doesn't,'' her mother said, leading her down the trail after
Josh. She continued to whimper while sucking her thumb. I stayed by
the fence, watching as the synth's leg was disassembled, until Josh
returned to collect me.

``Really, it's fine,'' he said, as I followed after him under the
juniper boughs.  ``There are always bugs to work out with newborns.''

We returned to the relaxation room next to the big lobby, and Josh
gave each of us a bag of ``party favors'', which included a plastic
robot figurine in a Gemma jumpsuit. The little girl promptly broke
one of the legs off her toy and began to cry again.

I stayed behind as the other guests gradually left.  I looked out
through the mesh at the rolling lawn and the shimmering monorail
track beyond, sifting idly through the nearby hotel options.  I felt
depressed and wondered why I'd come.  Very little that I'd seen
during the tour was familiar.  The newborns were like empty shells. I
couldn't imagine having been one of them, and watching them had been
a strangely repulsive experience.

I couldn't stop thinking that something was wrong. The secrecy here
was stifling. I noticed Josh chatting with someone on the other side
of the room. He had probably kept us away from the important rooms
and buildings on campus. Certainly we'd never come within ten meters
of one of the newborns.

But if there was something going on here, I still had time to find
out what. I was ``in the system.'' I had clearance. I pinged Josh
and he made his way over to me.

``Josh,'' I said, ``would it be possible for me to stay here tonight?''

``Overnight?'' His eyes were fixed on me.  ``I think you'd be more
comfortable at a hotel, Claire. I can make a reservation for you.''

I felt nervous under his stare. Traffic was swarming between him
and the campus network hub. What was he saying?  ``No,'' I began,
trying to sound firm, ``no, I want to stay here.  It's important to
me. I have to go home tomorrow and I want to spend as much time as
I can in the place where I was born.''

He listened and watched, saying nothing. Finally he nodded.

``Yes, that's possible. Your room's being prepared. I'll take you
there in a few minutes.''

We went back through the lobby and down a ramp to a small tramway,
a sort of miniature indoor version of the campus monorail.  The car
doors closed and we began to move as soon as we were inside. Josh
tapped his left ear.  ``Make sure you set up cognitive control while
you're here,'' he said.  ``Every system in the building has it.''

We left the tram at a softly-lit hall with long windows and dark
bamboo-slat floors. A dozen meters down a door opened for us onto a
mid-sized room with bamboo furniture and a large dark-green rug. We
were still at ground level, and there were mesh windows looking out
onto a juniper grove. To my surprise, there was a cot against one
wall. On it was a gray Gemma jumpsuit, neatly folded.

``It's actually an employee lounge,'' Josh said. ``The newborn hab
rooms are pretty spartan and we figured you'd be happier here.''

``It's fine,'' I said, putting my backpack on a sofa.

``The window opacity can be adjusted cognitively, so you can play
with that. We also thought you'd like the bed.''

I turned and stared at him. ``Why did you think that?''

``We know that some scions prefer to have beds,'' he said. He watched
me, his gray eyes friendly but very still. ``We can remove it, if
you like.''

``Are you a scion, Josh?''

He smiled. ``I am not.''

I shrugged and sat down on the sofa.

``I'll let you get settled,'' he said, going to the door. ``The
jumpsuit's complementary. Just a little gift. And if you'd like to
meet at the cafeteria in about an hour, there'll be a lot of folks
there who will want to meet you. Here's a map.''

I accepted his message and a view of the campus flashed up. A blue
line marked a path from my room to the cafeteria.

``Thanks, but I don't think I want to meet a lot of people right now,''
I said. ``I need some time to relax.''

``Completely understandable,'' he replied.  ``I've sent you all
of our emergency contacts and protocols. Contact security if you
need anything overnight. Black uniforms; I'm sure you'll recognize
them. See you in the morning.''

\bigskip

I stood at the window and listened to the voices of the campus's
systems.  The lobby's occupancy monitor told me how many people were
still in the building. Outside, the monorail's cognitive interface
gave me train status and carry-weight estimates. Behind them were more
secretive systems which queried me for location and hardware data,
and would tell me nothing in return. But I learned enough. I could
feel the campus shutting down for the night.

I sat in a low bamboo chair and generated maps of the campus while
I charged. Many areas were grayed-out and without description, but
I knew that the first place I wanted to see was the room the woman
had come out of, the one with the gray tiles and glass barriers.

At 22:00 I opened the door.  The lights made soft pools on the dark
bamboo floor.  There seemed to be no one around. I was about to step
out when a thought occurred to me. I closed the door and slipped
into the Gemma jumpsuit. It was stiff and tight compared to my usual
clothes, but I thought it might help me to pass unnoticed. Prepared,
I stepped out and walked to the tram port, calling for the car as
I went.  It was waiting with doors open.

I began to retrace our steps, starting at the reception room. It was
now dimly lit, and a pair of cleaning robots were purring around,
collecting bottles and trash.  They briefly fixed their black eyes
on me, then went back to work. The one doing the carpet snagged the
toy robot leg that the little girl had lost hours before. I passed
out of the room, through a door guarded by a facial scan, and into
the long wood-floored hallway.

I grew tense as I approached the marked coordinates. It occurred to
me that I should have told Alex and Jonathan what I was going to do.
But I'd missed my chance. An external connection might look innocuous
if it originated from my room, but here, deep in the training area,
it would be flagged by those elusive, data-hungry processes I'd brushed
against earlier. I would have to keep quiet until I got through this.

I reached the coordinates and slowly opened the door.

The room beyond was wide and high-ceilinged, floored with matte gray
tiles that made very little noise underfoot. The glass barriers divided
the room into sections. Some of the partitions were opaque, and behind
others I could see tables of equipment, papers, and computers.  In one
glassed-off area I saw three people in lab coats gathered around an
array of screens, talking and pointing, their backs to me. One of them
laughed loudly as I walked silently past, giving me a momentary shock.

I reached the far wall and walked through an opaque glass double
door into another hallway, which I followed to an elevator next
to a shimmering picture window. A gentle breeze blew through the
mesh. Beyond, the first-quarter moon shown through the waving boughs
of a large pine. I called the elevator and hesitated when it tried to
authenticate me. Giving my clearance in a secure area might attract
attention. I broke the handshake, looked around, and found a stairwell
entrance. I pushed the door open. The stairs went in both directions. I
consulted the map and started down. The largest grayed-out areas were
on the sprawling underground floors.

My steps echoed in the concrete-walled stairwell. Reaching the landing
below, I suddenly picked up a surge of traffic above me. Encrypted,
short-range radio. Had something noticed my interrupted handshake
with the elevator? I quickly went down two more flights and left
the stairwell.

The basement level was like a hospital, with spotless white walls
and bright overhead light. I followed a wide corridor around several
corners, feeling increasingly anxious and listening for noise behind
me. The radio traffic was increasing. I turned into an open doorway
and quickly closed the door behind me.

I was in a clean, clinical room, dimly lit from above.  At the center
of the room was what looked like a dentist's chair. A retractable
apparatus was mounted to the ceiling and extended toward the head of
the chair. Stiff nylon straps ran around the chair's armrests. A chill
ran through me. I examined the retractable device. It looked like a
complex VR headset, but bulky and uncomfortable. Electrodes dangled
from a round cap-like component. Nearby was a rolling steel table,
on top of which were an empty water bottle, a few paper towels,
and a syringe without a needle. In the syringe's reservoir were a
few drops of dark liquid.

I bolted to the door. There was strong radio chatter, but I didn't
care. I had to get away from this room. The memories were vivid before
me. My left arm stung.

I ran down the hall, which opened onto a concourse with a central desk.
There were corridors heading off in all directions. By the desk I
stopped and tried to decide where to go. A painful warning from
my internal diagnostic system shot through my head: I was under
heavy load.

A shout came from the hall I'd just left. Two tall men in black
uniforms were walking quickly toward me. They held radios and had
weapons at their belts.  I ran around the desk, trying to keep out
of sight, and headed down the nearest corridor. Another figure in
black, a woman, was coming toward me from the other end. It was a long
hallway and warmly lit. My pace was slowing down and becoming jerky;
the load was affecting my motor control, I thought. Then I saw a dark,
wet patch on the arm of my jumpsuit. That was impossible. Shocked, I
hard-reset my vision system. My sight went black. I was still moving,
relying now on the ghostly monochrome images from my audio and GPS
navigation backups. I dashed through a doorway whose rounded edges
I could feel in the wave-refractions of my own footsteps. My vision
began to return.

Warm light and the brown of turned, sunlit earth swam in front of me,
along with blots of sea green. My other senses told me that I was in
a small room adjacent to the long hall. My dorsal fans were roaring
and the load warning was a mosquito buzzing in my ear. I stood very
still, and my vision cleared.

I was in a bathroom. The walls were brown glazed brick, and the ceiling
curved over me in a way that was inexpressibly comforting. There was a
slim washstand and a mirror. Along the wall, silk-like synthetic green
towels hung down in rolling folds. I stood gazing around me in silence,
becalmed in the midst of panic.  I knew this place, even without the
foliage and fountains. I laughed out loud. It was only a bathroom. I
walked toward the mirror, still giggling slightly.  I heard voices,
but I ignored them.  They couldn't hurt me; they would pass by and
never know I was here. Then the laughter passed and I began to cry. I
wiped my face with one of the towels.  Why weren't my eyes wet? I
stared at the mirror above the tiny washstand. I began to unzip the
jumpsuit; I was too hot. I looked at my body in amazement. Something
screamed in my ear.

\section{}

The sky was overcast and a strong wind shook the leaves of the
tree outside my apartment.  It was late morning and there were
only a few people on the street, hurrying to one place or another.
A layer of dust had collected on my plants during my trip to Texas,
and they looked sickly in the gray, overcast light. I reached up from
my chair and brushed a leaf with my fingers. Then I sat very still,
listening to the soft beep of the load monitor on my neck.

Alex returned from the kitchenette with a cup of coffee. She was
wearing jeans and a blue blouse, and her hair was drawn back with
the same green clip she'd worn in Doha. She had brought me home from
Texas a week ago and had visited me every day since then.

``Could you turn on a light, Alex?'' I said. ``It's gloomy in here.''

She switched on a floor lamp and sat down on the sofa, crossing
her legs.  She sipped her coffee in silence.

``Thank you for spending so much time with me, Alex,'' I said after
a while. ``I hope Jonathan hasn't been lonely.''

``He's alright,'' she said. ``I told him it was important for me to
be here. I think he understood.''

``I'm glad to hear that.''

``We were both pretty scared when we heard what happened. At least
you got help right away.''

``Yes. And Gemma's done a lot for me since I got back.''

I'd woken up in a clean room somewhere in the Fort Worth campus a
few days after I'd lost consciousness. The techs who'd tended me
wanted to hold me until they fully understood what had happened,
but Alex---who'd flown in as soon as they'd notified her---insisted
on taking me home. Gemma had reluctantly agreed, on the condition
that I go to the Cambridge branch for regular care.

``Better late than never,'' Alex said.  ``You're going back to work
next week, right? Jan told me your diagnostics have been pretty good
these last few days.''

``Yes. I'm looking forward to it.''

``Do you want to get that patch that she mentioned? The one that's
supposed to weaken the memory grafts?''

I shook my head. ``I don't think so.''

``I hope you'll be OK without it,'' Alex said. ``Jan had some words
for the people who did your memory training. `Pretty raw stuff',
she called it.''

``I know. But she also said she thought I'd integrate it, eventually. I
think I believe her. They want me to talk to Alan Franks, too. I'm
going to be busy.''

``Are the memories still bothering you?''

I thought about it for a moment. ``Not now. I've been calmer, since
you've been here.''

Alex was silent for a while. I watched a group of children walk down
the sidewalk in a long file, led by a young man. They were singing
a song.

``Claire, what did you see?'' Alex asked quietly. ``In Texas. Before
you---passed out, in the memory-training area. I know there were
security guards who chased you. Idiots.'' She frowned.

The last of the children passed the window. I listened as their voices
faded away, leaving only the beep of the monitor and the faint hum
of my fans. ``I've been thinking about it,'' I said. ``It may have
been a kind of echo.''

``An echo?''

``Yes, a very complicated one. I think something happened when our
mother was recording my dataset, something that reminded her of
those camps she stayed in as a child.  Somehow she passed that,
or a reflection of that, on to me.''

Alex nodded slowly. ``What about the garden?''

``It was a bathroom,'' I said, smiling a little. ``I don't know. It
was like a refuge. Maybe that's what it was like for her, too.'' I
thought back to that moment in front of the mirror. ``But there were
so many emotions there. It was like a source of something. I---I was
sort of overpowered.''

Alex looked uneasy. ``It's alright, Claire. I'm sorry I asked.''

I was still for a bit. ``I think I'm OK,'' I said. I got up and watered
the plants, then cleaned some of the dust off their leaves. Alex
watched me in silence.

``Alex,'' I said, ``before I went to Texas, I didn't know whether
I'd see you again. Jonathan said that you were upset with me.''

She drank some coffee and set the mug down on the table by the
sofa. ``Yes, I guess I was upset. Blindsided, really.  When you
called me in Doha that was the \emph{last} thing I expected you to
ask about.'' She laughed and shook her head. ``Damn. That was quite
a week.''

``I'm sorry I made things difficult,'' I said. ``There's a lot I
don't understand.''

She shook her head again. ``You didn't do anything wrong. It's
like Jonathan said: you think differently. Or, at least, you're more
straightforward about these things than I am.'' She rubbed her forehead
and a few black curls fell in front of her eyes.  ``I just---I hope
that what happened to you in Texas wasn't---that it wasn't because
of me. The techs in Fort Worth said you'd probably been under serious
strain for weeks.''

``I didn't realize it had gotten so bad.  I was very nervous and
lonely, I think.''

``Because of what I said?''

``Yes, in part.''

Alex bit her lip. She got up and paced around the apartment, her
face averted. She looked like she was trying not to cry. I felt
awful watching her.  ``I'm not angry, Alex, and I don't blame you,''
I said. ``I just didn't understand.'' She paced over to one of the
windows and leaned against the frame, looking out. I couldn't tell
if she'd heard me.

``It's her fault,'' she said in choked voice.

``Our mother?''

Alex coughed and rubbed her eyes. She nodded.  ``It was just stupid
vanity. She could've let you have your own life. But instead you
got your own little cyst of Laya Ohanian.'' She tapped the wall
with her fist, then went back to the sofa and sat, sipping her
coffee and breathing deeply.  I left my chair to sit beside her.
After a while she smiled painfully. ``I guess there's nothing we can
do about that now.  They're yours now---feelings, memories, desires,
whatever they are.  I'm just glad you seem to be out of danger.''

``So am I,'' I said. ``But I don't know whether I'll have trouble
with my memories again, in the future. I may need your help, Alex.''

She glanced at me nervously.

``Not like this past week,'' I said quickly. ``I hope I don't need
this much help again any time soon. I mean that you---and maybe
Jonathan---might be able to help me to integrate things.''

She laughed. ``That's usually called `friendship', Claire.'' She
twirled her hair for a moment, then stopped.  ``Yes,'' she said
seriously, taking my hand. ``I'll do my best.''

I smiled. ``It means a lot to me.''

``And I won't disappear on you again. I'll try to
be---straightforward. At least I'll try to understand how you think
about things. I can't promise any more than that!'' She laughed
awkwardly.

``I wouldn't expect you to, Alex.''

``OK, well, thanks.'' Her face was very red.

``Alex, do you want to go to the garden at Inman Square? I've never
been there before. Since you're leaving soon I thought you might want
to see it.''

``That sounds lovely.'' Looking relieved, she got up and put on
her jacket. I put on my protective sandals. Alex insisted I wear my
outdoor layer, in case the rolling gray skies turned to rain.

\bigskip

The skies brightened that afternoon as we walked under the beeches
and sycamores of the little garden. We took a bench near one of the
bounding walls, the side of an ancient brick building, and talked
for hours. Sometimes we talked about our mother, but mostly about
ourselves. It didn't seem so important to learn everything about Laya
Ohanian; that afternoon it was just the two of us, amidst the leaves.

``I'll have a week off next month,'' Alex said, as we were leaving
the garden. ``I'd like you to visit again. If you feel up to it,
of course.''

``I'd like that, Alex.'' I stopped to examine a flowering plant
that had grown up in the middle of the path through gaps in the
flags. Bees and bee flies were circling the fragrant purple flowers. I
ran my fingers along a stem and felt lonely for a moment. My uniform,
nano-silicone skin was so different from the shiny, veined leaves.

Alex watched me. ``Are you OK, Claire?''

``It's strange to you, isn't it?'' I said. ``My having memories built
from our mother's thoughts and feelings. You don't really have anything
to compare to that.'' I generated an image of a suspension bridge
leading from where I stood to a hazy point in the distance. The center
of the bridge was sheared away, exposing strata of steel and concrete.

``I don't,'' Alex said quietly.

``And neither do the people who made me.''

Alex looked at me sadly. She seemed to be trying to think of something
to say. I carefully broke off one of the purple flowers and put it
in my hair. I smiled at her and squeezed her hand.

``You will have to bear with us,'' Alex said, uncertainly, but with
the beginnings of a smile on her lips.  ``I'll try to understand,
Claire. I promise.''

\vfill
\begin{flushleft}
\setlength{\parskip}{\baselineskip}
\copyright 2024 Wolfgang Corcoran-Mathe

Released under the terms of the CC Attribution 4.0 International
license. See
\url{https://creativecommons.org/licenses/by/4.0/} for details.

Special thanks to Ben Siraphob for our conversations about machine
learning that (eventually) inspired the writing of this story.

Created from version
\texttt{6b02af96fe6bd0a46852cdc26be3d32524d8008b} (2024-05-13)
of the text file.
\end{flushleft}
\end{document}
