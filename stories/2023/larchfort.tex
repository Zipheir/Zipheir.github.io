\documentclass[10pt,b5paper]{article}
\usepackage{geometry}
\begin{document}
\title{Larchfort}
\author{Wolfgang Corcoran-Mathe}
\date{}
\maketitle

The guards were still grumbling over the last of their breakfast
when Rysik stepped out into the gray light of sunrise. Blinking,
he brushed off his uniform, adjusted his newly-polished helmet,
and looped his belt through his scabbard. He was satisfied at being
the first guard out this morning, and glad to have gotten away from
the guardhouse's crowded mess hall, which was dark and hazy with
pipe smoke. He leaned against the door frame and surveyed the shabby
yard. Chicken bones, dog dung, and broken barrel-staves littered the
narrow grounds. One side was littered with lichen-covered stones,
shed by the crumbling city wall that loomed over the yard. Outside
the crooked fence, farmers and poor merchants were beginning to
move down Millet Street.

There was a clatter from inside the guardhouse. Rysik looked into
the dim vestibule, stroking his thin, fair beard nervously. The
guards came out in clumps. They were mostly young men and women
with strong bodies, loud voices, and peasant accents. Several were
red-faced and glassy-eyed. The last of them, a burly man with wild
eyes, tumbled into the yard with a gnawed mutton-leg. He tossed
it aside and gave Rysik a clownish grin. When Rysik ignored him,
the man made a rude gesture with his stubby fingers.

At last Corporal Menwith, tall and barrel-chested, emerged from the
guardhouse. He strode down the line, humming to himself. The guards
stood at attention, then leaned back against the fence as soon as
the corporal passed.

Menwith reached the end of the line. He nodded and opened the
gate. ``Out we go,'' he said. The guards filed into the street. As
Rysik passed, the corporal grabbed him by the shoulder.

``Did you think you'd get a bloody medal for being first out?''
he growled.

``No sir,'' said Rysik. ``I just thought I'd be timely today. For my
own good, mostly, but I thought it might lift the others' spirits,
too.'' He nodded at the guards, who were loitering outside the gate.

Menwith looked disgusted. He pushed his dented helmet up to scratch
his ear. ``You might raise someone's spirits acting like a bloody fop,
but it gives me a headache,'' he said.

``I heard some complaints, sir,'' Rysik persisted. ``Some say the
guardhouse is a cesspit, that we've gotten slovenly. I was trying
to help.''

The corporal stared at Rysik for a moment, his eyes narrowed and
his jowly face frowning. ```Help','' he said. ``You were trying to
help! Quite the young lord, ain't you, Rysik? Our reputation's been
worrying you, has it?'' He spat next to Rysik's boots.

``No!'' Rysik cried defiantly. ``But we won't get much respect from
folk who call us sluggards and gluttons.''

``Respect, eh. You don't fool me, Rysik. You think you're too bloody
good for us, is all. You set yourself above your fellow guardsmen
again and you'll be sorry. Understood?''

Rysik saluted stiffly. ``Understood.''

``Out,'' growled Menwith. Rysik marched through the gate. The corporal
followed with a slow, rolling stride. ``And no more nonsense on
duty,'' he said. ``It's latrine duty if you arrest any more vagrants
or drunken farmers. Leave the merchants alone, too.''

Rysik didn't reply. He and the other guards formed groups and set
out on their daily rounds.

Although the sun was not yet fully up, the streets were busy. It
was the third day in a week-long festival for the Baron's birthday.
Children in dirty linen were sweeping up confetti and trash in
the street. Several farmers were asleep against the wall of the
Fish-Wife, the tavern across the street from the guardhouse, and
one of the waiters was trying to awaken them by banging a pan with
a wooden spoon. A group of people in rumpled, gaudy festival robes
was gathered near the front door, laughing at the sleepy farmers.

Rysik grimaced. He hated the ugly tavern and its drunken crowds.
His bunk was on the first floor, and the singing and shouting last
night had made it impossible to sleep. As he marched past the
wide, squat, cheaply-whitewashed Fish-Wife, he cursed the place
and wondered whether he'd be able to stay awake for the day's
10-hour rounds.

\bigskip

Rysik returned to the guardhouse as the sun was setting. He got a
plate of cold chicken from the kitchen and went to his bunk to eat.
The barracks was quiet, but the revelry in Millet Street was just
beginning. Sitting on his straw mattress, Rysik watched the gathering
crowds from the window above his bunk.

Although Rysik avoided the tavern, his eye was drawn to the women
who frequented it on busy evenings. Recently, he'd noticed a young
woman with an erect posture who came early and seemed to keep to
herself. Her hair was always neatly braided and she wore clean
dresses. He wondered if he'd see her that evening. Rysik ate his
dinner and waited.

It was night by the time she arrived, and the street was now bathed
in the flickering light of torches and lanterns. She was wearing
a dark blue dress and ribbons, and was straggling behind a noisy
group of women headed for the tavern. One of the women called out
and beckoned to her from the front door. The young woman hesitated,
then disappeared inside.

Still hungry after a day of tramping around the city, Rysik went
back to the kitchen for a second helping. Then, exhaustion took
him and he dozed in his bunk for nearly an hour.

When he awoke, he saw the young woman in the blue dress wringing her
hands under the eaves of the tavern, some distance from the crowded
front door. As Rysik watched, a heavy man stumbled out of the tavern
and looked around, his red face shining in the torchlight. He strode
toward the young woman, waving his hands and shouting. She turned
and stepped quickly around the corner of the building into a narrow
street running along the side of the tavern. The man pursued, and
the young woman darted further into the dark alley. After a moment
of searching, the red-faced man stamped off after her into the gloom.

Rysik jumped out of the bunk and scrambled into his uniform. He ran
to the armory for a rapier, then out the door, the blade banging
against his knee as he tried to buckle it to his waist. Two guards
sitting in the yard looked up from their dice game and greeted him
tipsily. Ignoring them, he ran out of the gate into the crowd in
the street.

``Make way! City guard!'' he yelled. His voice sounded thin in the
uproar. Only a few people moved aside. It took him minutes of
shoving and squeezing to cross the street. The alley smelled of
stale ale, and beyond the glow of the street it was dark except for
a single green paper lantern burning some twenty yards ahead. In
the lantern light Rysik recognized the girl in the blue dress and
her pursuer. The heavy man faced the girl, his back to Rysik. One
of his hands was curled in a fist, while the other grasped the
girl's shoulder.

Rysik ran toward the light, drawing his rapier. ``City guard!''
he shouted. ``Hands off her!''

The man turned and squinted at the approaching guard. His broad
face was flushed and sweaty, with peevish, drunken eyes. His cloth
was expensive, though rumpled, and his pointed shoes were trimmed
with silver.

``Go on,'' the heavy man said. ``You a guard or a bandit, boy, sneaking
up like that? I wasn't harming the young lady. Clear off and leave
us alone.'' His eyes were sharp and his hand stayed on the young
woman's shoulder.

``Hands off, I said.'' Rysik's hand trembled as he grabbed the man's
puffy red sleeve with his free hand. There was no one else in sight,
and this wealthy-looking man was likely armed. He tried to keep
his voice level. ``Go home, sir. Leave the lady alone.''

``Don't touch me!'' the man roared, shaking Rysik off. ``Street scum
is what you are. Envious, greedy urchin in a man's uniform.'' He
stroked the girl's hair. ``Guard, my eye. I know what'll fix you.''
The man rummaged in his pockets and drew out something that gleamed.

Rysik struck with the flat of his blade. Copper and silver coins
showered on the dirt, and the man howled, rubbing his hand. The girl
looked ready to bolt, but Rysik was between her and the well-lit
street. She took a step to his right.

``Hold!'' said Rysik, more harshly than he had intended. ``Um, wait
a moment, miss.'' The girl stopped, glaring furiously at Rysik. Her
anger disconcerted him.

Muttering angrily, the heavy man bent to pick up his scattered money.
He stuffed the coins into a pocket and shook his unbruised fist at
Rysik. He was still rubbing his hand and tears were starting from
his eyes.

``You'll be sorry for this,'' he said. He looked at the young woman,
who was close to Rysik, trying to edge away toward the street.
His face twisted in rage.

``The two of you!'' he said. ``A trap! A bloody trap! And to think I
almost paid! Both of you! I'll fix you up for this, my street rats.''

Rysik stared at the man, taken aback. ``Nothing of the sort, sir,'' he
said stiffly. ``The festival gives you no license to molest a woman.''
He turned to the girl. ``Were you hurt, miss? Do you have a charge---''

``No, of course not!'' she cried. She turned away, wringing her hands.

``D'ye see now, you lout?'' shouted the man. The blow to his hand
had raised a thick welt.

``I see that I did well to interrupt when I did,'' Rysik said weakly.
``Since the lady appears unhurt and doesn't wish to make a complaint,
you may, um, go.''

``A fool and a common thug, that's what you are. Go off with your
slut. You'll hear about this, my boy.'' The heavy man looked quickly
around in the dirt for any missing coins, spat at Rysik's feet,
and strode off toward the street.

As soon as he was gone, the girl turned on Rysik and shoved him
forcefully in the shoulder. Rysik tripped backward. She was strong
and slightly taller than he was.

``He's right, you are a fool,'' she said. ``A cut-purse, too. I nearly
made something of that before you fouled things. I haven't had a
penny all evening.''

Her voice, though harsh with anger, had an educated accent. Her
dress, though well-made, was patched and dusty, and her fine features
were exaggerated by undernourishment. A loose lock of straight
brown hair fell along her cheek. Her brown eyes were unfocused,
from fatigue or perhaps drink.

``I saw you run from him,'' Rysik protested, ``He was drunk and meant
to harm you.''

``He was nothing I couldn't handle,'' she said. ``D'you think you're
the only one who carries a blade in case of trouble?'' She patted
her skirt pocket. ``He was a boor, and as handsy as the rest of the
fine lads in there.'' She gestured at the tavern. Her face was pained.

Rysik nodded. ``A rough place, to be sure.'' He sheathed his blade
and looked sheepish.

Worry seemed to consume the girl. ``What'll he do?'' she said,
more to herself than to Rysik. ``Kellyn is a big customer, they told
me. If he thinks I tried to catch him in a game and tells the keep,
well \ldots'' Her anger returned and she turned back to Rysik. ``I don't
believe a guard could be so stupid as to beat a man for talking to
a---a whore, and then come out with nonsense like `the lady is unhurt'
like he was in the baron's court. You're some stable-boy or street
actor who stole a uniform and went off clowning. I should take you
to the guards and let them beat sense into you.''

Annoyed, Rysik fished out his round bronze badge and offered it to
her. ``I have served as a guardsman of the Third Ward for six months
and have been considered for distinction twice,'' he said. (This was
an exaggeration: Corporal Menwith's considerations had been purely
sarcastic.) ``And I must warn you, Miss: meeting that drunkard back
here, armed as you are, makes his charge sound more likely. Your
work is legal, but \ldots''

``You're a fool,'' she repeated, and began to walk back toward Millet
Street. She seemed tired and preoccupied. She limped slightly;
Rysik noticed that her right shoe had come apart and was bound
together with string. Pity stung him, and he regretted his reproof.

``Can I, er, may I escort you to a safe place?'' he said, following
her.

``Get away from me. Find another girl to insult.''

``I meant no disrespect! Perhaps you came to the festival with
your family?''

``I have no family in Larchfort,'' she replied and curtseyed
sarcastically. ``But my thanks for your kind offer, Sir Clown. I do
hope you will excuse me and find your own way back to your circus
tent. Good night!'' She reached the street, turned, and went into
the tavern.

He walked back to the guardhouse, weaving his way through the
crowds and feeling dismal. A reveler threw his arm around him and
tried to get him to sing a song. Rysik freed himself, laughing
weakly. He got back to the barracks, undressed, and crawled into
his bunk. The party in the street roared on. He tossed and turned,
and it was a long time before he fell asleep.

\bigskip

The festival continued for the next three days, during which time
Rysik saw nothing of the young woman. He did his rounds and came
back exhausted and depressed each evening. His mind returned again
and again to the argument in the dark alley.

On the day after the celebrations had ended, Corporal Menwith
beckoned Rysik into the guardhouse's armory. His big face was grey
and his eyes were bloodshot from the previous night's revelry. He
sighed, sat down heavily in a stout oak chair, and pointed to a
stool next to the grinding wheel. Rysik sat.

``You've been busy, Rysik,'' Menwith said in a tired voice. ``Got a
complaint today from that silk merchant, Kellyn. Says you tried to
game him during the festival, then beat him when he wouldn't pay.''

Rysik's face felt hot. He scratched his beard. ``It isn't true,''
he said.

``'Course it isn't true,'' Menwith said. ``It was more of your damned
foolishness, I'd wager. But that doesn't make an ounce of difference.
He's a rich man and has a lot of pull in the city. He wants you
kicked into the street.''

``But he's a drunk and a lecher!'' Rysik said, glaring at the
corporal. ``He was holding a woman against her will, and I did my
duty. There were no other guards around to stop him, either---at
least none that could hold a weapon.''

``Watch your tongue,'' growled the corporal. ``I'm the only one between
you and the gutter, so you'll show some respect. Kellyn also says
that woman was part of the game, that you two were going to split
the takings after you nabbed him.'' He sighed again. ``He's going to
make life difficult for her, too.''

``Damn him,'' said Rysik. ``She did nothing wrong. I made a mistake,
that's all.''

Menwith leaned back, looking up at the ceiling with a sardonic grin.
``By the gods, he's learning! That you did, and I told him you'll be
punished. Half pay and latrine duty for the next three months. Kellyn
wasn't pleased with that, but it'll take him a bit of trouble to
do more.'' He stared at Rysik. ``Go to a temple and pray sometime,
boy. You made a powerful enemy, and you'll need all the help you
can get.'' Menwith pushed himself up out of the chair. ``Dismissed.''

Rysik stood and saluted. ``Thank you, sir.''

Menwith nodded brusquely and went out.

\bigskip

It was another three days before he saw the young woman again,
sitting on the rim of a well far from Millet Street. It was drizzly
midday, and she was trying to keep dry under the well's small peaked
roof. Rysik recognized her dress as the blue one she'd worn the
week before, but it was now damp and muddy.

He hailed her as he approached. She turned away when she saw him.

``Miss!'' he said, brushing the rain and road-grit off his face. ``Where
have you been? I worried that something had happened to you, when
I didn't see you on Millet Street after last week.''

She laughed mirthlessly. ``Something did happen. I was barred from
the Fish-Wife.''

``Barred?'' said Rysik. ``By who?''

``The keep, Jospur. Said I'd been gaming his customers. He would've
barred you too if he'd known you, but it seems your lordship doesn't
frequent such vulgar places.''

It was starting to rain in earnest, and Rysik's sodden helmet
was weighing on him. He pulled it off and sat down on the well,
keeping his head under the roof. The hollow beating of the rain on
the roof-boards echoed in the well.

``And why should you?'' he said indignantly. ``It's a filthy barn. I
can't imagine why you'd want to come within a mile of the place,
let alone--- That is, well---if you had a choice.'' His anger subsided
into miserable uncertainty.

She nodded vaguely, watching the street. A pair of horses were
plodding by in front of an old cart. The driver huddled under a wet,
ragged cloak.

``I, er, hope you have someplace to go?'' Rysik said.

``One of the girls is lending me a cot,'' she said. ``The one who
told me to try whoring. I don't think that'll last, though, if
I can't earn anything. She says I botched it at the Fish-Wife.''
The girl smiled bitterly. ``She says I meant to botch it. And she
doesn't know yet that I can't go back. She only saw me leave early
and she nearly took it out of my hide.''

Rysik saw loneliness, fear, and a kind of miserable triumph in her
eyes. He reached up uncertainly and placed his gloved hand on her
shoulder. She batted it away listlessly.

``I'm sorry,'' he said. ``But there must be something better.'' He
thought for a moment. ``You told me you carried a blade. Would you
join the guard? If you can fight, you'd beat most of our lot.''

She nodded and her look softened a bit. ``I can fight,'' she said.
She drew a thin-bladed dagger from her pocket and unsheathed it.
She rubbed it on her skirts, then put it away. ``I had to learn,
back when I was selling things on the streets.''

``Very well.'' Rysik stood up and put his helmet back on. ``I'll speak
to Corporal Menwith tonight. He'll want to test you, and you'll have
to learn rapier and halberd. You'll live with a bunch of drunken
louts, but at least you won't have to hang on any of them to earn
your supper.'' He smiled grimly.

She grunted. ``For a fire and regular meals I can live with it.''

Rysik nodded and began to walk off down the street. He turned back
to the young woman, who was still sitting with her arms crossed
under the little roof.

``What name should I give? To the corporal,'' he said.

She looked at him dully for a moment. ``Lenyrl.''

``Lenyrl,'' Rysik repeated. He trudged off through the mud. She had
lost a chance at making a living through his mistake, he thought,
and now he would find her another. He did the rest of his rounds
hopefully, but with a trace of fear. His life in the guardhouse
was barely tolerable; would it be any better for Lenyrl?

\bigskip

Rysik told Menwith about Lenyrl that evening. The corporal
immediately refused to enlist her. Any friend of Rysik's, he said,
was bound to cause trouble for the guard. Rysik was moody during
dinner, but his fortunes improved. After eating several large
servings of mutton soup with roasted pumpkin and downing a small
keg of ale, Menwith became gracious. He agreed to test the young
woman the following day.

When he entered the barracks after dinner, Rysik was met with
raucous cheers. By this time, the guards had heard of his mishap
with Lenyrl and Kellyn. Tally, the wild-eyed, burly guard who was
fond of mutton, grabbed Rysik around the shoulders and laughed in
his face. His uniform was badly stained and smelled of brandy.

``Rysik's got himself a tart!'' he shouted to the guards. There
was another cheer. Tally leered at Rysik. ``You're a naughty lad,
though, using her as bait. That horny merchant fondles her and you
let him go for a bit of coin, eh? No, you put her to better uses.''
The guards roared and stamped their feet. Tally waggled his finger at
Rysik theatrically. ``Stay away from them old tricks, young lad. You
might get yourself arrested.''

Rysik smiled weakly and pulled away. Another guard, a tall man who
had sat next to Rysik at table, clapped Tally around the shoulders
and whispered in his ear. Tally's eyes opened wide.

``By the gods, Rysik's got old Menwith to let his tart onto the
guard!'' He cackled. ``Clever lad! Won't that be a treat! I hope
he don't mind sharing. Drink, lad! Drink to our new tart, and to
you what's brought her to us.'' Tally picked up a glass and thrust
it forward.

Rysik's face burned. He slapped the drink out of Tally's hand
and shoved the stout man out of his way, making for his bunk.
The guards hooted.

``Ooo, he's jealous!'' Tally cried.

The tall guard looked disgusted. ``That fool should learn not to
fall in love with whores. This house'll go to hell if he behaves
like that when she's here.''

Tally nodded. He strode over and leaned his grinning bearded face
over Rysik. ``Once a whore, always a whore, Rysik!'' he shouted. The
other guards yelled in agreement, the women as much as the men.

Rysik lay on his bunk and fumed. There was no point in trying
to tell his fellow guards the truth, that Lenyrl had made only
a failed attempt at prostitution. They would, of course, prefer
their version. He began to feel nauseous as he thought of how they'd
described Lenyrl, with her thin face and once-neat blue dress. There
could be no life for her here. She could not be a guard; they would
only accept her as a barracks whore.

A guard in the upper bunk roared out in laughter and began to tell
a joke at the top of his lungs. Rysik kicked the slats above him
with both boots, and his neighbor responded by throwing his own
boot at Rysik's head. It struck him in the face. Tears welled up,
and Rysik hid his face in his pillow.

He had to apologize, he thought, if nothing else. And he would
still try to help her, if he could.

\bigskip

The next day he made excuses to the hungover Menwith. The corporal
had forgotten about Lenyrl, and was more confused than annoyed. Rysik
saw the young woman during his rounds, and told her that he could
not get her a place on the guard. She shrugged and said nothing.

As he walked, Rysik grimly resolved to try to persuade Jospur, the
keeper of the Fish-Wife, to let Lenyrl back in. He hated the idea.

When he got back to Millet Street it was nearly sundown, though
little of the sun could be seen through the gray mass of clouds
gathered over Larchfort. The rain had come and gone as Rysik walked
his rounds, and he was now thoroughly damp and mud-spattered.
Under the eaves of the Fish-Wife, he dumped small rocks from his
wet boots and tried to clean his beard with some trough water. This
done, he went into the tavern.

The Fish-Wife was quiet, with only a few laborers sitting at the
long, dark tables. The big rectangular room smelled of tallow and
stale ale. It was a dingy, battered box, poorly decorated by faded
tapestries and banners. A boy in an apron was scattering new straw
on the floors, taking occasional sips from a cup which he held in his
left hand. Bengt Jospur was standing near the bar, arguing with a man
in a leather cap who was towing a cart of flour sacks. The portly,
full-bearded owner glanced at Rysik when the young guard came in. He
returned Rysik's greeting with a hand raised against interruption.

Stifled for the moment, Rysik ordered small beer, cheese, and bread
from the straw-boy. The bread was coarse and black and the cheese
old, but Rysik was hungry. After a few minutes, the provisioner
unloaded his goods and left, grumbling. Jospur sat on the stool
next to Rysik and peered at him with watery brown eyes.

``Evening, guardsman,'' he said in a low, rough voice. ``Fare
satisfactory, I hope?''

Behind his solicitous manner, Jospur was impatient. Rysik swallowed
and tried to sit up. He would have to make his case quickly.

``A woman has been barred from this place, sir,'' he said. ``A young
woman, named Lenyrl, who came here during the festival. I've come
to request that she be allowed to return.''

Jospur thought for a moment. ``Lenyrl. Pale girl, braids, blue dress
with patches? She came in with the other girls last week.''

Rysik nodded.

Jospur's face hardened. ``No. By the gods, she's a thief. Kellyn
the silk merchant, one of the biggest men around here, said she
meant to bleed him. She sat right here and told me she was some
kind of whore-in-training, but I know her type. Sharp, and good
with a blade, maybe.''

``Sharp but honest, sir,'' said Rysik. ``Kellyn tried to force himself
on her. Can you not see that he means to hide his disgrace, and
revenge himself on this girl?''

``That's as may be,'' Jospur said. ``What's your word against his? I've
never seen you before.'' He rubbed his small, red nose and went on.
``Kellyn told me a bit more. Said that this Lenyrl had a friend in the
guards who came up on him while he was talking with her. The fellow
beat him, then went off with the girl. Sort of a trap, you know?''

Rysik shifted uneasily and said nothing.

Jospur continued. ``Kellyn didn't know this guard, but, from his
description, you ain't far from the mark.''

``I was that guard,'' Rysik said after a moment. ``I struck Kellyn once,
because he wouldn't release the woman. It was rash, and I've been
punished. But the rest of his story's a lie.''

``A lie, eh? Maybe, maybe.'' Jospur shrugged, beginning to look bored.

``The young woman must be allowed back, even if I am not,'' Rysik
went on. ``Her living depends on it. You have my word that she'll
pursue it honestly.''

``Maybe,'' Jospur said again. ``It ain't worth my time to find out. All
I know is Kellyn's a good customer and I ain't crossing him, not for
a street rat and a boy guard.'' He grunted as he stood up. ``No, lad,
she'll not be back. As for you, you'll stay away from Kellyn if you
know what's good for you. I won't have you driving away business.''
He gave a curt nod and walked off.

Rysik ate the rest of his meal in silence. He could think of no way
of changing Jospur's mind. The man wasn't interested in the facts.
Rysik's fist clenched as he watched Jospur stride about the tavern,
ordering his staff here and there and rubbing boot marks from the
benches. Lenyrl, he thought, should not have to depend on this
lout's good opinion for her living. Rysik dropped some coins on
the bar and walked out.

\bigskip

Rysik met Lenyrl by the well again the next day. She listened
stoically as he described his conversation with Jospur. He suggested
other jobs for her---at a jeweler's shop near Millet Street, or
\ldots\@ She
shrugged and thanked him quietly. He didn't see her again that week.

His failure to help Lenyrl weighed on him. As the days passed, he
felt as if he were slipping into a quagmire. Though the guards no
longer teased him about Lenyrl at every opportunity, the guardhouse
had become intolerable. He ate little, and began to wander around
the city in the evenings, returning to the barracks just before
curfew to scrub the latrine. He dozed through his rounds. When
Menwith threatened him with dismissal, Rysik shrugged and promised
to do better, without conviction.

One night, in a deeper gloom than usual, he walked all the way to
the cobbled, sloping streets of the high town, where the sound of
his steps echoed off the tall, dark houses. It was past midnight,
and he thought wearily that Menwith would punish him for staying out
after curfew. But he could not turn back. His path wound around and
around the district, which was deserted except for the occasional
beggar or wanderer. He imagined Lenyrl walking the same path,
just out of sight beyond the next bend.

The moon rose as the bells rang in their somber tower, and Rysik
looked out over the silent city from the top of a steep road. He
saw a maze of shadowed wood and stone, encircled far off by the
city wall. The rolling land beyond seemed impossibly distant.
The streets of Larchfort trapped him, and would, he thought, hold
him and Lenyrl for the rest of their lives.

It was dawn when he finally returned to Millet Street. Rysik's heart
sank when he saw an armed figure in a heavy cloak waiting near the
guardhouse door. He had hoped that his return would be unnoticed,
and now prepared himself for a tongue-lashing.

The figure waved to him. It was Lenyrl. She was wearing a leather
breastplate and an old-fashioned helmet. Her dagger was in her belt,
and she wore muddy traveling boots over moth-eaten wool stockings.

``I was about to give up and leave,'' she whispered as he came into
the yard. ``Where were you?''

Rysik was speechless for a moment. ``In the high town,'' he said,
more loudly than he'd meant to. She put a finger to her lips.
``Why are you here?''

``I need a sword,'' she said. She tapped her dagger. ``This was enough
for Larchfort, but I'm done with this rubbish heap.'' She looked
at Rysik, and a grin broke across her face. ``Sir Clown,'' she said,
``would you do me the honor of nicking me a blade?''

``A sword?'' he said.

She frowned impatiently and punched his shoulder with her mittened
fist. ``Yes! Now, before they're awake!''

Rysik did as she asked. He took a short sword that had hung in
the armory for as long as he could remember, then went to the
kitchen. He wrapped some leftover food in a cloth and returned to
the yard. There was no sound from the barracks.

Lenyrl thanked him quickly and buckled the sword to her belt. She
wrapped the thick wool cloak around her, concealing the weapons.

``Where will you go?'' asked Rysik.

``Somewhere better,'' Lenyrl said. ``I'll follow the merchant route
out of town. I can do mercenary work. Maybe I'll look for my family.''

``Mercenary work? I can't imagine why---'' For a moment Rysik looked
scandalized; then he sighed and nodded. ``I understand. Please
be careful.''

``Yes, your lordship,'' she said, scowling and putting the food in a
small quilted bag. ``Thanks for the grub. Maybe you ought to leave,
too, when you're done trying to act like a lord in a pig sty.''

Rysik thought of the raucous barracks and the red-eyed guards,
and felt a powerful longing to leave and be rid of them. He stared
into Lenyrl's pale face, which now looked strong and confident.

``I couldn't,'' he said sadly. ``I've made a mess of things here,
and I'd do as much on the road or anywhere else. You're right,
I've been playing at being a lord when---''

``Bottle it up,'' she said. ``Crying about it is worse than doing
nothing.'' She shrugged, and Rysik saw disappointment in her
eyes. ``You did try your best, you know,'' she said, with a thin smile.

Rysik stood mutely for a moment. ``Thanks.''

``Well then, I'm off.'' She went out of the yard, gave Rysik a
brief wave, and set off down Millet Street toward the city gate,
her cloak flapping in the breeze.

Rysik watched her pass through the gate, then stood in the
yard rubbing his tired eyes and watching the sun rise above the
wall. After a while, he went inside, where a few guards were now
stirring. Rysik drew a mug of small beer and sat at the long table by
himself, deep in thought. After a few minutes the guards filed in,
and the room was filled with the smell of sweat and bacon fat. They
ate and went off to the armory. Soon came Menwith's mustering call,
and the hall was full of the sound of hobnailed boots and clanking
arms. Rysik continued to sip at his cup and think of Lenyrl's
cloaked figure disappearing down the road.

When he finally stood and prepared to leave, he had made a
decision. As she had found a way out and taken it, so would he. He
would find something better than his dull rounds and the sordid
dealings of Larchfort. He would follow Lenyrl; if not today,
then soon.

\vfill
\begin{flushleft}
\setlength{\parskip}{\baselineskip}
\copyright 2023 Wolfgang Corcoran-Mathe

Released under the terms of the CC BY 4.0 license. See
https://creativecommons.org/licenses/by/4.0/ for details.

This file was created from version
\texttt{6b2a9032b6923ddee7d59a74f1bb7af0f64e5f5e} (2024-03-06) of
the plain text file.
\end{flushleft}
\end{document}
